Caminhos metabólicos definem coletivamente o repertório bioquímico de um organismo assim como os métodos pelos quais esse organismo produz seus produtos químicos. Os caminhos metabólicos têm sido reconstruídos usando uma larga variedade de métodos computacionais e fonte de dados, incluindo dados  {\it -omics} e restrições tais como a presença de enzimas, estequeometria e termodinâmica. A reconstrução de cada caminho em relação ao outro pode variar por alguns aspectos, como a granularidade dos detalhes bioquímicos. A demonstração de reações químicas utilizando a reescrita de grafos é um método que possibilita apresentar os produtos iniciais, finais e intermediários por meio da modelagem dos caminhos como hiperfluxos inteiros. Neste trabalho, foram esboçadas as reescritas de grafos as quais descrevem reações cíclicas catalisadas pela sintase dos terpenos das plantas para produzir monoterpenos, sendo também gerado o espaço químico possível para explorar o potencial da sintase de monoterpenos. Nesse sentido, é fornecida uma interface gráfica com objetivo de facilitar a criação dos códigos de simulação.

\begin{keywords}
	caminhos metabólicos, monoterpenos, biossíntese, plantas, reescrita de grafos
\end{keywords}